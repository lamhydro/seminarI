% !TEX encoding = UTF-8 Unicode
\documentclass[
10pt,
aspectratio=169,
]{beamer}
\setbeamercovered{transparent=10}
\usetheme[
%  showheader,
%  red,
  purple,
%  gray,
%  graytitle,
  colorblocks,
%  noframetitlerule,
]{Verona}

\usepackage[T1]{fontenc}
\usepackage[utf8]{inputenc}
\usepackage{lipsum}
%%%%%%%%%%%%%%%%%%%%%%%%%%%%%%%
% Mac上使用如下命令声明隶书字体,windows也有相关方式,大家可自行修改
%\providecommand{\lishu}{\CJKfamily{zhli}}
%%%%%%%%%%%%%%%%%%%%%%%%%%%%%%%
\usepackage{tikz}
\usetikzlibrary{fadings}
%
%\setbeamertemplate{sections/subsections in toc}[ball]
%\usepackage{xeCJK}
\usepackage{adjustbox} % Shrink stuff
\usepackage{listings}
\usepackage{caption}
\usepackage{subcaption}
\usefonttheme{professionalfonts}
\def\mathfamilydefault{\rmdefault}
\usepackage{amsmath}
\usepackage{multirow}
\usepackage{booktabs}
\usepackage{bm}
\setbeamertemplate{section in toc}{\hspace*{1em}\inserttocsectionnumber.~\inserttocsection\par}
\setbeamertemplate{subsection in toc}{\hspace*{2em}\inserttocsectionnumber.\inserttocsubsectionnumber.~\inserttocsubsection\par}
\setbeamerfont{subsection in toc}{size=\small}
\AtBeginSection[]{%
	\begin{frame}%
		\frametitle{Outline}%
		\textbf{\tableofcontents[currentsection]} %
	\end{frame}%
}

\AtBeginSubsection[]{%
	\begin{frame}%
		\frametitle{Outline}%
		\textbf{\tableofcontents[currentsection, currentsubsection]} %
	\end{frame}%
}

\title{B\'usqueda y revisi\'on bibliografica}
\subtitle{Herramientas para la elaboraci\'on de la propuesta}
\author[L.M.]{Luis Alejandro Morales, Ph.D.}
\mail{lmoralesm@unal.edu.co}
\institute[UNAL]{Facultad de Ingenier\'ia, Departamento de Ingnenier\'ia Civil y Agr\'icola\\
Universidad Nacional de Colombia, Bogot\'a}
\date{\today}
\titlegraphic[width=3cm]{logo_01u}{}

%%%%%%%%%%%%%%%%%%%%%%%%%%%%%%%%
% ----------- 标题页 ------------
%%%%%%%%%%%%%%%%%%%%%%%%%%%%%%%%
% New commands
\newcommand{\gi}{\texttt{Git}}
\newcommand{\gih}{\texttt{GitHub}}
\newcommand{\co}[1]{\alert{\textbf{\large \texttt{#1}}}}
\begin{document}



\maketitle

%%% define code
\defverbatim[colored]\lstI{
	\begin{lstlisting}[language=C++,basicstyle=\ttfamily,keywordstyle=\color{red}]
	int main() {
	// Define variables at the beginning
	// of the block, as in C:
	CStash intStash, stringStash;
	int i;
	char* cp;
	ifstream in;
	string line;
	[...]
	\end{lstlisting}
}
%%%%%%%%%%%%%%%%%%%%%%%%%%%%%%%%
% ----------- FRAME ------------
%%%%%%%%%%%%%%%%%%%%%%%%%%%%%%%%

%---
\section{Manejadores de referencias}

\begin{frame}[c]{Importancia de los manejadores de referencia}
\begin{itemize}
\item Incrementa la productividad.
\item Organiza la bibliografia y disminuye la posibilidad de plagio.
\item Mejora la calidad de las citas en el texto.
\item Permite la toma de notas sobre documentos consultados.
\item Almacenar PDF de art\'iculos. Hace la busqueda de estos. 
\item Intuitivos y faciles de usar.
\end{itemize}
\end{frame}


\begin{frame}[c]{Manejadores de referencia}
Existen diferentes manejadores de referencia. Los m\'as comunes son:
\begin{table}
\begin{adjustbox}{width=\textwidth}
\begin{tabular}{ l c c c }
& \textbf{Disponibilidad} & \textbf{Compatibilidad} & \textbf{Otros}\\
 \href{https://www.zotero.org/}{\alert{Zotero}} & Free. Web-based \& desktop & MS Word, OpenOffice \& Google docs & Popular\\
 \href{https://www.mendeley.com/}{\alert{Mendeley}} & Free. Web-based \& desktop & MS Word \& OpenOffice & Popular \\
 \href{https://endnote.com/es/}{EndNote} & Lisenced. Web-based \& desktop (independent) & MS Word & M\'as sofisticado \\
 \href{https://refworks.proquest.com/}{RefWorks}: & Lisenced. Web-based & MS Word \& Google docs & Large databases \\
 \href{https://www.citavi.com/en}{Citavi} & Lisenced. Web-based \& desktop (independent) & MS Word \& Google docs & Popular in Germany 
\end{tabular}
\end{adjustbox}
\end{table}
\end{frame}

%---
\section{Lectura eficiente}
\begin{frame}[c]{Lectura activa y eficiente}
La lectura activa:
\begin{columns}
\column{0.6\textwidth}
\begin{itemize}
\item Es selectiva
\item Es critica
\item Interactua con el texto
\item Cambia el orden de lectura
\item Se lee de nuevo con un proposito
\item Anticipa
\end{itemize}
\column{0.4\textwidth}
\centering
\includegraphics[width=\textwidth]{linus.jpeg}
\end{columns}
\end{frame}

\begin{frame}[c]{Pasos}
\centering
\includegraphics[width=\textwidth]{linus.jpeg}
\end{frame}

\begin{frame}[c]{Seleci\'on}
Preguntas a formularse antes y durante la lectura
\begin{itemize}
\item ¿Aceptacion, diseminaci\'on?
\item ¿Credibilidad?
\item ¿Relevancia?
\item ¿Que es nuevo para mi?
\item ¿Que conozco ya?
\item ¿En que orden debo leer las secciones/cap\'itulos?
\item ¿Que necesita mayor atenci\'on?
\end{itemize}
\end{frame}


\begin{frame}[c]{Organizaci\'on}
Tomar notas y organizarlas de acuerdo con:
\begin{itemize}
\item Palabras claves
\item Parafrasear ideas
\item Comparaci\'on con otros trabajos
\item Opinar con criterio
\item Formular preguntas sobre el texto
\item Mapeo de la informaci\'on
\end{itemize}
\end{frame}

\begin{frame}[c]{Aceleracion}
Problemas en la velocidad de lectura:
\begin{itemize}
\item Subvocalizaci\'on
\item Vuelta atras
\item Interrupciones
\item Luz baja e incomodidad
\item Fatiga y cansansio
\item Vocabulario pobre y mala comprension
\item Sesiones largas de lectura
\end{itemize}
Curas simples:
\begin{itemize}
\item Mejorar la postura y comodidad
\item Buena luz
\end{itemize}
\end{frame}


\begin{frame}[c]{¿Porque la copia impresa es mejor?}
\begin{itemize}
\item Las anotaciones son flexibles y faciles
\item Navegar el documento es mas facil
\item Es mas recomentable cuando se hacen multiples tareas
\item Facil para manejar/consultar varios documentos
\end{itemize}
\end{frame}

\begin{frame}[c]{Retenci\'on}
\alert{SMART} reading:
\alert{S}pecific \alert{M}apped \alert{A}chievable  \alert{R}elevant \alert{T}ime-limited 
Pasos en el proceso de lectura:
\begin{enumerate}
\item Mapeo y revisi\'on
\item Vista anticipada
\item Escaneo, ojeado: Anticipar la estructura y el contenido.
\item Lectura y notas: Preparar tu sitio de trabajo, alcance de la lectura, materiales, etc.
\item Revision y reflexion
\end{enumerate}
\end{frame}


\begin{frame}[c]{Concentraci\'on y comprensi\'on}
\begin{table}
\begin{adjustbox}{width=\textwidth}
\begin{tabular}{ c c }
\textbf{Concetraci\'on} & \textbf{Comprensi\'on}\\
Determinar objetivos especificos & Mejorar tu vocabulatrio (prefijos, sufijos, terminos tecnicos) \\
Programacion real lectura & Remover distracciones \\
Leer puntual & Reducir fijaciones por linea \\
Leer en el mismo sitio & 
\end{tabular}
\end{adjustbox}
\end{table}
¿Que se retiene?
\begin{itemize}
\item Conceptos significativos e importantes
\item Conceptos familiares o unicos
\item Conceptos faciles de mapear
\end{itemize}
\end{frame}


\begin{frame}[c]{¿Como trabaja la memoria y el cerebro?}
La memoria se mejora:
\begin{columns}
\column{0.4\textwidth}
\begin{itemize}
\item Atenci\'on e inter\'es
\item Organizaci\'on 
\item Practica
\end{itemize}
\column{0.6\textwidth}
\centering
\includegraphics[width=\textwidth]{linus.jpeg}
\end{columns}
\end{frame}


\begin{frame}[c]{Retenci\'on}
Como mejorarla:
\begin{itemize}
\item Reestablecer y limpiar la \alert{memoria de trabajo}
\item Asociar la lectura usando la \alert{memoria transitoria} y las nuevas asociaciones creadas
\item Permitir interacciones mas libres con la \alert{memoria de largo plazo}
\end{itemize}
\end{frame}

%---
\section{Revision de la literatura}
\begin{frame}[c]{¿Que es la revisi\'on de la bibliograf\'ia?}
\alert{Una evaluación organizada y crítica de publicaciones para responder una pregunta.}
\alert{No es simplemente un catalogo descriptivo de publicaciones.} 
\alert{Por lo tanto, es importante conocer y entender la pregunta de investigaci\'on.}
\end{frame}
%---

%
\section{Plagio}






%\section{?`Que es \gi?}
%\begin{frame}[c]{?`Que es \gi?}
%\begin{columns}
%\column{0.6\textwidth}
%\large{
%\begin{block}{\gi}
%\begin{itemize}
%    \item \gi\ es un programa/sistema usado para el control de versiones en proyectos, particularmente, en c\'odigos de computador.
%    \item \gi\ fue inventado en 2005 por Linus Torvalds, creador de \texttt{Linux}, con el fin de manejar proyectos grandes multiusuario (e.g. kernel de \texttt{Linux} escrito en \texttt{C}) de manera eficiente y r\'apida.
%    \item Git est\'a escrito en \texttt{C} y viene por defecto en \texttt{Linux}. Puede ser instalado en \texttt{MS Windows}.
%\end{itemize}
%\end{block}
%}
%\column{0.4\textwidth}
%\begin{center}
% %\begin{figure}
% \includegraphics[width=\textwidth]{linus.jpeg}
% %\vspace{-0.5cm}
% %\caption{\tiny Fuente: https://en.wikipedia.org/wiki}
% %\end{figure}
% \end{center}
%\end{columns}
%\end{frame}	
%%---
%\section{?`Para que sirve \gi?}
%\begin{frame}[c]{?`Para que sirve \gi?}
%%\begin{columns}
%%\column{0.6\textwidth}
%\large{
%\begin{block}{Utilidades de \gi}
%\begin{itemize}
%    \item Trabajo en grandes proyectos con colaboracion de multiples usuarios y servidores.
%    \item Control de cambios lo que permite ir a hacia adelante o hacia atras en la historia del proyecto.
%    \item Sistema distribuido lo que permite que multiples usuarios hagan cambios y sean a la vez servidores del proyecto.
%\end{itemize}
%\end{block}
%}
%\end{frame}	
%
%\begin{frame}[c]{Modelo centralizado Vs \alert{modelo distribuido}}
%\vspace{-0.2cm}
%\begin{center}
% \includegraphics[width=0.87\textwidth]{dist.png}
% \end{center}
%\end{frame}	
%%---
%\section{Tratamiento de cambios en un repositorio en \gi}
%\begin{frame}[c]{Proyecto local en \gi}
%\vspace{-0.2cm}
%\begin{center}
% %\begin{figure}
% \includegraphics[width=0.55\textwidth]{gitp.png}
% %\vspace{-0.5cm}
% %\caption{\tiny Fuente: https://en.wikipedia.org/wiki}
% %\end{figure}
% \end{center}
%%\end{columns}
%\end{frame}	
%
%\begin{frame}[c]{Ciclo de vida de un archivo de un repositorio en \gi}
%%\column{0.4\textwidth}
%\begin{center}
% %\begin{figure}
% \includegraphics[width=0.8\textwidth]{gitf.png}
% %\vspace{-0.5cm}
% %\caption{\tiny Fuente: https://en.wikipedia.org/wiki}
% %\end{figure}
% \end{center}
%%\end{columns}
%\end{frame}	
%
%%---
%\section{\gih}
%\begin{frame}[c]{\gih}
%\begin{columns}
%\column{0.8\textwidth}
%\begin{block}{\gih}
%\begin{itemize}
%    \item \href{http://www.github.com}{\gih} es una p\'agina web de libre acceso para archivar repositorios online.
%    \item Muchos repositorios de codigo abierto como el kernel de Linux usan \gih.
%    \item ?`Es necesario tener \gih\ para usar \gi? \textbf{No!}
%    \begin{itemize}
%        \item Se puede usar \gi\ localmente, o  
%        \item se puede configurar un servidor para compartir archivos.
%    \end{itemize}     
%\end{itemize}
%\end{block}
%\column{0.2\textwidth}
%\begin{center}
% \includegraphics[width=0.8\textwidth]{ghub.png}
% \end{center}
%\end{columns}
%\end{frame}
%
%
%%---
%\section{Commandos b\'asicos en \gi}
%\begin{frame}[c]{Workflow b\'asico en \gi}
%\begin{block}{Workflow b\'asico en \gi}
%\begin{enumerate}
%    \item \alert{Pull} el directorio \gi\ del servidor remoto (opcional).
%    \item \alert{Modificar} los archivos en el directorio de trabajo
%    \item \alert{Stage} archivos. Adicionar un copia del archivo a la staging area.
%    \item Hacer un \alert{commit}, el cual toma los archivos en la staging area y archiva la copia en el directorio de \gi. 
%    \item \alert{Push} el directorio \gi\ a el servidor remoto (opcional).
%\end{enumerate}
%\end{block}
%\begin{center}
% \includegraphics[width=\textwidth]{workflow.png}
%\end{center}
%\end{frame}	
%
%\begin{frame}[c]{Configuraci\'on inicial en \gi}
%\begin{block}{Alistarse para usar \gi}
%    \begin{enumerate}
%    \item Introducir \texttt{username} y \texttt{email} para ser usado por \gi\ cuando se haga un \texttt{commit}.\\
%     \co{\$ git config --global user.name “Luis Morales”}\\ %\vspace{0.3cm}
%     \co{\$ git config --global user.email lmoralesm@unal.edu.co}
%     \begin{itemize}
%         \item Para verificar la configuraci\'on:\\
%              \co{\$ git config -list}
%         \item Esta configuraci\'on es para todos los proyectos \gi.
%         \item La configuraci\'on anterior puede ser para un proyecto determinado si no se usa la opci\'on \co{-{}-global}
%         \item Adem\'as en la configuraci\'on inicial se puede escoger el editor para escribir mensajes en commit:
%         \co{\$ git config -{}-global core.editor vim}
%     \end{itemize}
%    \end{enumerate}
%\end{block}
%\end{frame}	
%
%\begin{frame}[c]{Crear una copia local del repositorio}
%\begin{block}{Crear copia local del repositorio \gi}
%\begin{enumerate}
%    \item Dos escenarios posibles:
%    \begin{enumerate}
%        \item Clonar un repositorio existente en su directorio local:\\
%        \co{\$ git clone <url> [local\_dir\_name]}\\
%        Crea el directorio \texttt{local\_dir\_name}, el cual contiene una copia de los archivos del repo original, y un directorio \texttt{.git}.
%        \item Crear un repositorio \gi\ en el directorio actual:\\
%        \co{\$ git init}\\
%        Crea el directorio \texttt{.git} en el directorio actual.  
%    \end{enumerate}
%\end{enumerate}
%\end{block}
%\end{frame}	
%
%\begin{frame}[c]{Commit archivos}
%\begin{block}{Commit archivos en el repo local}
%\begin{enumerate}
%    \item Una ves creado el directorio, se puede:
%    \begin{enumerate}
%        \item Se pueden crear archivos dentro del repositorio y adicionarlos a la staging \'area: \\
%        \co{\$ git add README.md file1.c}\\
%        Toma un snapshop de estos archivos en este instante de tiempo y los adiciona a la staging \'area.
%        \item Commit cambios en el repo (mover los staged cambios al repo)
%        \co{\$ git commit -m "initial project version"}\\
%    \end{enumerate}
%    Para unstage cambios (deshacer cambios) en un archivo antes de commit este):\\
%    \co{\$ git reset HEAD -{}- filename}
%    Para deshacer cambios en un archivo despu\'es de commit este:\\
%    \co{\$ git checkout -{}- filename}
%\end{enumerate}
%\end{block}
%\end{frame}	
%
%\begin{frame}[c]{Status y Diff}
%\begin{block}{Status y Diff}
%\begin{itemize}
%    \item Para mirar el \texttt{status} de los archivos en el directorio de trabajo o repo y la staging area:\\
%    \co{\$ git status} o \\
%    \co{\$ git status -s}\\
%    donde la opci\'on \co{-s} muestra una versi\'on del archivo en la staging \'area.
%    \item Para ver cual archivo ha sido modificado pero no est\'a en la staging \'area:\\
%    \co{\$ git diff}
%    \item Para ver cambios que ya estan en la staging \'area:\\
%    \co{\$ git diff}
%\end{itemize}
%\end{block}
%\end{frame}	
%
%\begin{frame}[c]{Chequeando los \texttt{logs}}
%\begin{block}{Chequeando los \texttt{logs}}
%\texttt{log} es un comando en \gi\ que permite conocer los cambios hechos en el repo. Algunos comandos importantes:
%\begin{itemize}
%    \item Chequeando la versi\'on larga de \texttt{logs}:\\
%    \co{\$ git log}
%    \item Chequeando la versi\'on larga de \texttt{logs}:\\
%    \co{\$ git log -{}-oneline}
%    %\begin{center}
%    %\includegraphics[width=\textwidth]{}
%    %\end{center}
%    \item Para mostrar los 5 m\'as recientes cambios:\\
%    \co{\$ git log -5}
%\end{itemize}
%Notas:
%\begin{enumerate}
%    \item Los cambios son listado de acuerdo con el commitID \#.
%    \item Todos los cambios hechos en el repo antes de haber sido clonado o jalado estan incluidos en \texttt{logs}.  
%\end{enumerate}
%\end{block}
%\end{frame}	
%
%\begin{frame}[c]{Pulling y pushing}
%\begin{block}{Pulling y pushing}
%\texttt{pull} y \texttt{push} son dos comandos que permiten jalar el repo de un servidor externo y enviar cambios a el repo, respectivamente. Buenas pr\'acticas en el uso de estos comandos son:
%\begin{enumerate}
%    \item \texttt{Add} y \texttt{commit} los cambios al repo local.
%    \item \texttt{pull} del repo remoto para obtener los cambios mas recientes. En caso de conflictos, \texttt{add} y \texttt{commit} estos al repo. 
%    \item \texttt{push} los cambios al repo remoto.
%\end{enumerate}
%Para incluir los cambios m\'as recientes del repo remoto en el repo local:\\
%\begin{enumerate}
%    \item Para bajar el contenido del repo remoto al repo local (optional):\\
%    \co{\$ git fetch}
%    \item Para bajar el contenido del repo remoto y actualizar el repo local:\\
%    \co{\$ git pull origin master}\\
%\end{enumerate}
%Para \texttt{push} los cambios realizados en el repo local a el repo remoto:\\
%\co{\$ git push origin master}
%\end{block}
%\end{frame}	
%
%\begin{frame}[c]{Branching}
%\vspace{-0.3cm}
%\begin{block}{Branching}
%\texttt{branch} es un comando que permite crear ramificaciones dentro del repo local para hacer cambios experimentales en el.
%\begin{itemize}
%    \item Para crear un \texttt{branch} llamado e.g. experiment1:\\
%    \co{\$ git branch experiment1}
%    \item Para listar todos los branches en el repo:\\
%    \co{\$ git branch}\\
%    Note que * indica el branch actual
%    \item Para cambiar al branch \texttt{experiment1}:\\
%    \co{\$ git checkout experiment1}
%    \item Para introducir los cambios hechos en  \texttt{experiment1} dentro del branch \texttt{master}:\\
%    \co{\$ git checkout master}\\
%    \co{\$ git merge experiment1}      
%\end{itemize}
%Notas:
%\begin{enumerate}
%    \item \co{\$ git log -{}-graph} puede ser usado para mostrar los branches gr\'aficamente. 
%    \item Los branches est\'an solo en el repo local.
%\end{enumerate}
%\end{block}
%\end{frame}	
%
%\begin{frame}[c]{Resumen}
%\vspace{-0.2cm}
%\begin{block}{Resumen}
%Configuraci\'on inicial y clonaci\'on
%\begin{enumerate}
%    \item \co{\$ git config -{}-global user.name “Your Name”}
%    \item \co{\$ git config --global user.email youremail@unal.edu.co}
%    \item \co{\$ git clone https://github.com/hydsrg/hyds-repo.git}
%\end{enumerate}
%Editar y visualizar cambios \texttt{hyds-repo}
%\begin{enumerate}
%    \item \co{\$ git log}; \co{\$ git log -{}-oneline}
%    \item Create a file, e.g. \texttt{filename.txt}
%    \item \co{\$ git status}; \co{\$ git status –s}
%    \item Agregar el archivo al repo (staging area): \co{git add filename.txt}
%    \item \co{\$ git status}; \co{\$ git status –s}
%    \item \texttt{commit} el archivo al repo local:\\
%    \co{\$ git commit –m “added filename.txt file”}
%    \item \co{\$ git status}; \co{\$ git status –s}; \co{\$ git log -{}-oneline}
%\end{enumerate}
%\end{block}
%\end{frame}	
%
%\begin{frame}[c]{Resumen}
%\begin{block}{Resumen}
%Pulling y pushing los cambios
%\begin{enumerate}
%    \item \texttt{pull} de un repo remoto: \co{\$ git pull origin master}
%    \item \texttt{push} hacia un repo remoto: \co{\$ git push origin master}
%\end{enumerate}
%\end{block}
%\end{frame}	

%
\end{document}


