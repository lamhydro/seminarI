% !TEX encoding = UTF-8 Unicode
\documentclass[
10pt,
aspectratio=169,
]{beamer}
\setbeamercovered{transparent=10}
\usetheme[
%  showheader,
%  red,
  purple,
%  gray,
%  graytitle,
  colorblocks,
%  noframetitlerule,
]{Verona}

\usepackage[T1]{fontenc}
\usepackage[utf8]{inputenc}
\usepackage{lipsum}
%%%%%%%%%%%%%%%%%%%%%%%%%%%%%%%
% Mac上使用如下命令声明隶书字体,windows也有相关方式,大家可自行修改
%\providecommand{\lishu}{\CJKfamily{zhli}}
%%%%%%%%%%%%%%%%%%%%%%%%%%%%%%%
\usepackage{tikz}
\usetikzlibrary{fadings}
%
%\setbeamertemplate{sections/subsections in toc}[ball]
%\usepackage{xeCJK}
\usepackage{adjustbox} % Shrink stuff
\usepackage{listings}
\usepackage{caption}
\usepackage{subcaption}
\usefonttheme{professionalfonts}
\def\mathfamilydefault{\rmdefault}
\usepackage{amsmath}
\usepackage{multirow}
\usepackage{booktabs}
\usepackage{bm}
\setbeamertemplate{section in toc}{\hspace*{1em}\inserttocsectionnumber.~\inserttocsection\par}
\setbeamertemplate{subsection in toc}{\hspace*{2em}\inserttocsectionnumber.\inserttocsubsectionnumber.~\inserttocsubsection\par}
\setbeamerfont{subsection in toc}{size=\small}
\AtBeginSection[]{%
	\begin{frame}%
		\frametitle{Outline}%
		\textbf{\tableofcontents[currentsection]} %
	\end{frame}%
}

\AtBeginSubsection[]{%
	\begin{frame}%
		\frametitle{Outline}%
		\textbf{\tableofcontents[currentsection, currentsubsection]} %
	\end{frame}%
}

\title{B\'usqueda y revisi\'on bibliografica}
\subtitle{Herramientas para la elaboraci\'on de la propuesta}
\author[L.M.]{Luis Alejandro Morales, Ph.D.}
\mail{lmoralesm@unal.edu.co}
\institute[UNAL]{Facultad de Ingenier\'ia, Departamento de Ingnenier\'ia Civil y Agr\'icola\\
Universidad Nacional de Colombia, Bogot\'a}
\date{\today}
\titlegraphic[width=3cm]{logo_01u}{}

%%%%%%%%%%%%%%%%%%%%%%%%%%%%%%%%
% ----------- 标题页 ------------
%%%%%%%%%%%%%%%%%%%%%%%%%%%%%%%%
% New commands
\newcommand{\gi}{\texttt{Git}}
\newcommand{\gih}{\texttt{GitHub}}
\newcommand{\co}[1]{\alert{\textbf{\large \texttt{#1}}}}
\begin{document}



\maketitle

%%% define code
\defverbatim[colored]\lstI{
	\begin{lstlisting}[language=C++,basicstyle=\ttfamily,keywordstyle=\color{red}]
	int main() {
	// Define variables at the beginning
	// of the block, as in C:
	CStash intStash, stringStash;
	int i;
	char* cp;
	ifstream in;
	string line;
	[...]
	\end{lstlisting}
}
%%%%%%%%%%%%%%%%%%%%%%%%%%%%%%%%
% ----------- FRAME ------------
%%%%%%%%%%%%%%%%%%%%%%%%%%%%%%%%

%---
\section{Escritura en ciencias}

\begin{frame}[c]{Generalidades}
\begin{itemize}
\item Citaciones y menciones a su trabajo de tesis es lo mas importante.
\item Una tesis es buena cuando tiene gran influencia y aceptacion.
\item Cuando los evaluadores y colegas entienden el trabajo.
\item Lo anterior se logra escribiendo de una forma efectiva.
\item Es bueno seguir autores en su campo y tratar de emular su escritura.
\end{itemize}
\end{frame}

\begin{frame}[c]{Generalidades}
\begin{itemize}
\item Para escribir claro se debe pensar claro.
\item Una buena escritura, permite transmitir ideas claramente haciendo el trabajo mas exitoso.
\item Es importante desarrollar el habito de escribir (e.g. 1 hora diaria). No esperar a tener e.g. graficas.
\item Al final, el trabajo consiste en hacer el trabajo de lectura facil.
\item Para que la escritura sea efectiva, conviertase en un escritor.
\end{itemize}
\end{frame}


\begin{frame}[c]{Escritura vs re-escritura}
\begin{itemize}
\item Entienda que sus primeros borradores no los vera nadie, asi que no importa como escriba. 
\item Intentar escribir un primer borrador de forma perfecta puede ser paralizante.
\item Los mejores escritores intentan escribir su primer borrador rapidamente. Solo para tener algo que revisar.
\item Siempre sera mucho mas facil revisar que escribir de cero. 
\item Note que re-escribir es la esencia de la escritura.
\item La escritura es un proceso iterativo sobre varias versiones de un texto (e.g. 10 versiones). 
\item Este proceso iterativo garantiza claridad en el texto. 
\end{itemize}
\end{frame}


\begin{frame}[c]{Ejercicio}
Intente escribir 5 min sin parar sobre su idea de tesis. 
\end{frame}

%---

\section{Contar una historia}

\begin{frame}[c]{Ejercicio}
\begin{itemize}
\item El problema con los cientificos es que no saben contar historias.
\item Un texto cientifico no solo presenta unos datos, los interpreta. 
\item Un texto construye una historia a partir de los resultados.
\item Aquellos textos que son capaces de contar una buena historia son los que mas tienen impacto.

\end{itemize}
\end{frame}
%---
\section{Hacer que la historia sea atractiva}
%---
\section{Estructura de la historia}




\end{document}


