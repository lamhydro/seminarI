% !TEX encoding = UTF-8 Unicode
\documentclass[
10pt,
aspectratio=169,
]{beamer}
\setbeamercovered{transparent=10}
\usetheme[
%  showheader,
%  red,
  purple,
%  gray,
%  graytitle,
  colorblocks,
%  noframetitlerule,
]{Verona}

\usepackage[T1]{fontenc}
\usepackage[utf8]{inputenc}
\usepackage{lipsum}
%%%%%%%%%%%%%%%%%%%%%%%%%%%%%%%
% Mac上使用如下命令声明隶书字体,windows也有相关方式,大家可自行修改
%\providecommand{\lishu}{\CJKfamily{zhli}}
%%%%%%%%%%%%%%%%%%%%%%%%%%%%%%%
\usepackage{tikz}
\usetikzlibrary{fadings}
%
%\setbeamertemplate{sections/subsections in toc}[ball]
%\usepackage{xeCJK}
\usepackage{adjustbox} % Shrink stuff
\usepackage{listings}
\usepackage{caption}
\usepackage{subcaption}
\usefonttheme{professionalfonts}
\def\mathfamilydefault{\rmdefault}
\usepackage{amsmath}
\usepackage{multirow}
\usepackage{booktabs}
\usepackage{bm}
\setbeamertemplate{section in toc}{\hspace*{1em}\inserttocsectionnumber.~\inserttocsection\par}
\setbeamertemplate{subsection in toc}{\hspace*{2em}\inserttocsectionnumber.\inserttocsubsectionnumber.~\inserttocsubsection\par}
\setbeamerfont{subsection in toc}{size=\small}
\AtBeginSection[]{%
	\begin{frame}%
		\frametitle{Outline}%
		\textbf{\tableofcontents[currentsection]} %
	\end{frame}%
}

\AtBeginSubsection[]{%
	\begin{frame}%
		\frametitle{Outline}%
		\textbf{\tableofcontents[currentsection, currentsubsection]} %
	\end{frame}%
}

\title{B\'usqueda y revisi\'on bibliogr\'afica}
\subtitle{Herramientas para la elaboraci\'on de la propuesta}
\author[L.M.]{Luis Alejandro Morales, Ph.D.}
\mail{lmoralesm@unal.edu.co}
\institute[UNAL]{Facultad de Ingenier\'ia, Departamento de Ingnenier\'ia Civil y Agr\'icola\\
Universidad Nacional de Colombia, Bogot\'a}
\date{\today}
\titlegraphic[width=3cm]{logo_01u}{}

%%%%%%%%%%%%%%%%%%%%%%%%%%%%%%%%
% ----------- 标题页 ------------
%%%%%%%%%%%%%%%%%%%%%%%%%%%%%%%%
% New commands
\newcommand{\gi}{\texttt{Git}}
\newcommand{\gih}{\texttt{GitHub}}
\newcommand{\co}[1]{\alert{\textbf{\large \texttt{#1}}}}
\begin{document}



\maketitle

%%% define code
\defverbatim[colored]\lstI{
	\begin{lstlisting}[language=C++,basicstyle=\ttfamily,keywordstyle=\color{red}]
	int main() {
	// Define variables at the beginning
	// of the block, as in C:
	CStash intStash, stringStash;
	int i;
	char* cp;
	ifstream in;
	string line;
	[...]
	\end{lstlisting}
}
%%%%%%%%%%%%%%%%%%%%%%%%%%%%%%%%
% ----------- FRAME ------------
%%%%%%%%%%%%%%%%%%%%%%%%%%%%%%%%

%---
\section{Escritura en ciencias}

\begin{frame}[c]{Generalidades}
\begin{itemize}
\item \alert{Citaciones} y \alert{menciones} a su trabajo de tesis es lo m\'as importante.
\item Una tesis es buena cuando tiene gran \alert{influencia} y \alert{aceptaci\'on}.
\item Cuando los evaluadores y colegas entienden el trabajo.
\item Lo anterior se logra \alert{escribiendo} de una forma \alert{efectiva}.
\item Es bueno seguir \alert{autores} en su campo y tratar de \alert{emular} su escritura.
\end{itemize}
\end{frame}

\begin{frame}[c]{Generalidades}
\begin{itemize}
\item Para escribir claro se debe pensar claro.
\item Una buena escritura, permite transmitir ideas claramente haciendo el trabajo m\'as exitoso.
\item \alert{Es importante desarrollar el h\'abito de escribir (e.g. 1 hora diaria). No esperar a tener e.g. graficas.}
\item Al final, el trabajo consiste en hacer el trabajo de lectura f\'acil.
\item Para que la escritura sea efectiva, hay que convertirse en un escritor.
\end{itemize}
\end{frame}


\begin{frame}[c]{Escritura vs re-escritura}
\begin{itemize}
\item Entienda que sus primeros borradores no los ver\'a nadie, as\'i que no importa como escriba. 
\item Intentar escribir un primer borrador de forma perfecta puede ser paralizante.
\item Los mejores escritores intentan escribir su primer borrador r\'apidamente. Solo para tener algo que revisar.
\item \alert{Siempre ser\'a mucho mas f\'acil revisar que escribir de cero.}
\item \alert{Note que re-escribir es la esencia de la escritura.}
\item La escritura es un proceso iterativo sobre varias versiones de un texto (e.g. 10 versiones). 
\item Este proceso iterativo garantiza claridad en el texto. 
\end{itemize}
\end{frame}


\begin{frame}[c]{Ejercicio}
Intente escribir 5 min sin parar sobre su idea de tesis. 
\end{frame}

%---

\section{Contar una historia}

\begin{frame}[c]{Generalidades}
\begin{itemize}
\item El problema con los cient\'ificos es que no saben contar historias.
\item Un texto cient\'ifico no solo presenta unos datos, los interpreta. 
\item \alert{Un texto construye una historia a partir de los resultados.}
\item Aquellos textos que son capaces de contar una buena historia son los que m\'as tienen impacto.
\item Para contar una buena historia tu debes evaluar los resultados y determinar las posibles explicaciones.
\item Los personajes de nuestras historias son por ejemplos, rios, cuencas, lagos, etc
\end{itemize}
\end{frame}

\begin{frame}[c]{Encontrando una historia}
\begin{itemize}
\item Los m\'as importante no son los resultados, es el conocimiento que derivamos de ellos.
% include figure 2.2
\item Nuestra funci\'on es recolectar unos datos, transformarlos en informaci\'on, luego en conocimiento para luego ser sintetizado y usado para entender.
\item Inicie con los datos, anal\'iselos, escuche la historia que estos quieren transmitir, luego encuentre una historia.
\item No es recomendable adoptar la primera historia que venta a la cabeza. Es necesario explorar los limites y las fronteras de sus datos para encontrar la mejor historia. 
\end{itemize}
\end{frame}

\begin{frame}[c]{Ejercicio}
Escoja un art\'iculo de inter\'es y:
\begin{itemize}
\item Identifique los puntos importantes de la historia 
\item ¿Es f\'acil identificar la historia que se quiere contar?
\item ¿Podr\'ia haberse contado una mejor historia a partir de los resultados?
\end{itemize}
\end{frame}

%---

\section{Hacer que la historia sea atractiva}

\begin{frame}[c]{Generalidade}
\begin{itemize}
\item ¿Cuanto demora en olvidar lo que lee?
\item La t\'ecnica \alert{SUCCESS} hace que una idea perdure y no se olvide f\'acilmente:
\begin{itemize}
\item \alert{S}imple
\item \alert{U}nexpected
\item \alert{C}oncrete
\item \alert{C}redible
\item \alert{E}motional
\item \alert{S}tories
\end{itemize}
\end{itemize}
\end{frame}

\begin{frame}[c]{\alert{S}imple}
\begin{itemize}
\item Una idea que "pega" tiende a ser simple.
\item \emph{Simple} es diferente de \emph{Simplistic}
\item \alert{Entre mas simple una idea en su esencia, mayor impacto tiene.}
\item Una idea simple es aquella que describe el centro del problema.
\item Una idea simple se puede expresar en una ecuaci\'on, en una frase, en un dibujo.
\end{itemize}
\end{frame}

\begin{frame}[c]{\alert{U}nexpected}
\begin{itemize}
\item Un proyecto de investigaci\'on cuyo prop\'osito sea llenar un vac\'io, tiende a olvidarse.
\item Un buen trabajo de investigaci\'on es aquel que dice algo nuevo, algo \emph{inesperado}.
\item Su trabajo es destacar eso novedoso o inesperado de manera clara en el escrito.
\item Resaltando lo desconocido, se crea expectativa y se atrapa al lector gracias a su curiosidad.
\item Identificar un vac\'io en el conocimiento crea curiosidad. Llenar ese v\'acio es novedoso.
\end{itemize}
\end{frame}

\begin{frame}[c]{\alert{C}oncrete}
\begin{itemize}
\item Un mensaje simple es potente, pero un mensaje concreto es aun m\'as.
\item La ciencia vive en constante tensi\'on entre los datos (concretos) y entre ideas abstractas (contrario a algo concreto)
\item Vinculando una idea a una ejemplo concreto, convierte esa idea en algo concreto.
\end{itemize}
\end{frame}

\begin{frame}[c]{\alert{C}redible}
\begin{itemize}
\item Ciencia que no es creíble es ciencia fiction.
\item \alert{Nuestras ideas se vuelven cre\'ibles cuando tienen unas bases a partir de trabajos previos.}
\item Datos son cre\'ibles cuando describimos los m\'etodos usados en su procesamiento. Cuando se usan estadísticas apropiadas.
\item Conclusiones son cre\'ibles cuando estas surgen de datos cre\'ibles.
\end{itemize}
\end{frame}

\begin{frame}[c]{\alert{E}motional}
\begin{itemize}
\item La buena ciencia se hace siendo objetivo; sin pasiones.
\item La \'unica \alert{emoci\'on} admitida es la \alert{curiosidad}.
\item \alert{Entusiasmo} es la segunda \alert{emosion} admitida. Nace de la curiosidad.
\end{itemize}
\end{frame}

\begin{frame}[c]{\alert{S}tories}
\begin{itemize}
\item \alert{Una historia nace de la colecci\'on de peque\~nas historias enlazadas.}
\item Para escribir una tesis, es necesario pensar en la estructura interna del texto y como integrar esas peque\~nas historias.
\end{itemize}
Los seis componentes de \alert{SUCCES} son parte integral para contar una historia de una manera efectiva.
\end{frame}


%---
\section{Estructura de la historia}
\subsection{P\'arrafos}
\begin{frame}[c]{Generalidades}
\begin{itemize}
\item La unidad fundamental de un texto es el \alert{p\'arrafo}.
\item Las palabras son a los \'atomos como las frases a las mol\'eculas. El p\'arrafo es la c\'elula. 
\item Los pa\'rrafos cuentan historias con una estructura coherente.
\end{itemize}
\end{frame}

\begin{frame}[c]{Estructura de un p\'arrafo: \alert{"Topic Sentence-Development" (TS-D)}}
Consiste en introducir la idea principal del p\'arrafo en la primera frase. A partir de esta, se desarrolla el p\'arrafo. Esta estructura es simple y debe ser la estructura principal de los p\'arrafos escritos en su documento. En algunos casos, la idea introductoria se extiende a m\'as de una frase. 
\end{frame}

\begin{frame}[c]{Estructura de un p\'arrafo: \alert{"Point-Last Paragraphs" (P-LP)}}
Algunas veces es necesario ensamblar un argumento a trav\'es del enlace de ideas terminando en una conclusi\'on. Para este caso, un paragr\'afo puede iniciar con una idea de apertura (no necesariamente un argumento) y luego desarrollarlo el cual luego finaliza con una s\'intesis. Tanto el inicio como el final del p\'arrafo son fuertes. 
\end{frame}

\begin{frame}[c]{P\'arrafos mal escritos}
P\'arrafos que tienen una estructura incoherente son confusos y no transmiten un mensaje. Usualmente estos p\'arrafos pueden ser largos y sin un prop\'osito o idea clara. Este tipo de p\'arrafos ocurren cuando los autores saben mucho pero escriben poco, esto hace hace que las ideas se condensen mucho y sean desordenadas. 
Como arreglar un p\'arrafo malo:
\begin{enumerate}  
\item Identificar la historia que se quiere contar. 
\item Determinar que estructura es mas adecuada: TS-D o P-LP
\item Tomar las frases en el p\'arrafo para aclarar sus relaciones. 
\end{enumerate}  
P\'arrafos cortos son preferibles a p\'arrafos largos.
\end{frame}

\begin{frame}[c]{Ejercicio}
Del articulo que ha estado analizando, tome tres p\'arrafos de inter\'es y:
\begin{itemize}
\item Definir la estructura del p\'arrafo: TS-D o P-LP
\item Determinar la idea principal.
\end{itemize}
\end{frame}


\end{document}


